\documentclass{beamer}

\mode<presentation> {
  \usetheme{Warsaw}
  \setbeamercovered{transparent}
}

\usepackage{ucs}
\usepackage[utf8x]{inputenc}
\usepackage[czech]{babel}
\usepackage{palatino}
\usepackage{graphicx}
\usepackage{animate}

\begin{document}
\setbeamertemplate{caption}{\insertcaption}

\title[Tvorba aplikace pro analýzu výstupních dat z Flow123D]{Tvorba aplikace pro analýzu výstupních dat z~transportních procesů získaných ze softwaru Flow123D}
\author[Tomáš Křížek]{{\large Tomáš Křížek}\\{\scriptsize Vedoucí práce: Josef Chudoba, Ph.D.}}
\institute[TUL]{Technická univerzita v Liberci}
\date{17.~června~2014}

\begin{frame}
	\titlepage
\end{frame}

\begin{frame}
	\frametitle{Úvod do problematiky}
	\begin{block}{Flow123D}
		\begin{itemize}
			\item Hlubinné ukládání vyhořelého jaderného paliva.
			\item Horninový masiv reprezentován sítí elementů.
			\item Simulace transportních procesů látek do okolí.
			\item Výstupní data
			\begin{itemize}
				\item textové soubory (řádově jednotky GB)
				\item koncentrace látek na elementech v čase
			\end{itemize}
		\end{itemize}
	\end{block}
\end{frame}

\begin{frame}
	\frametitle{Cíle práce}
	\begin{block}{Software pro generovaní grafů a map}
		\begin{itemize}
			\item Vhodné načtení vstupních dat z textových souborů.
			\item Vygenerování \textbf{grafů} -- vývoj koncentrace v čase na daném elementu.
			\item Vygenerování \textbf{map} -- grafické zobrazení koncentrací na elementech v daném řezu sítě.
			\item Zpracování a porovnání sady úloh.
		\end{itemize}
	\end{block}
	\begin{block}{Motivace}
		\begin{itemize}
			\item Reprezentace dat ve formátu, kterému člověk porozumí.
		\end{itemize}
	\end{block}
\end{frame}

\begin{frame}
	\frametitle{Ukázka grafů}
	\hspace*{-10mm}\makebox[\paperwidth][c]{%
		\begin{minipage}{0.5\paperwidth}
			\centering
			\includegraphics[width=0.5\paperwidth]{img/graf.pdf} \\
			{\tiny Graf jedné úlohy}
		\end{minipage}%
		\begin{minipage}{0.5\paperwidth}
			\centering
			\includegraphics[width=0.5\paperwidth]{img/graf_vice_uloh.pdf} \\
			{\tiny Graf sady úloh}
		\end{minipage}
	}
\end{frame}


\begin{frame}
	\frametitle{Ukázka mapy}
	\centering
	\includegraphics[height=0.85\textheight]{img/mapa.png}
\end{frame}

\begin{frame}
	\frametitle{Použité technologie a knihovny}
	\begin{itemize}
		\item \textbf{Python 2.7} -- programovací jazyk
		\item \textbf{Matplotlib 1.3.1}
			\begin{itemize}
				\item knihovna pro generování grafů a map
				\item podpora exportu ve formátech rastrové i vektorové grafiky
			\end{itemize}
		\item \textbf{SQLite 3}
			\begin{itemize}
				\item databáze pro ukládání dat
				\item vzhledem k formátu vstupních dat je vhodnější použít databázi než seznamy či pole
				\item umožňuje uložení databáze pouze v paměti po dobu běhu programu
			\end{itemize}
		\item \textbf{PyQt 4} -- grafické rozhraní
	\end{itemize}
\end{frame}

%\begin{frame}
%	\frametitle{Ukázka animace mapy}
%	\animategraphics[autoplay,loop,height=0.85\textheight]{2}{img/v2/melechov-}{1}{9}
%\end{frame}

\begin{frame}
	\frametitle{Výsledky práce}
	Aplikace, která
	\begin{itemize}
		\item je schopná načíst data z transportních procesů získaných z Flow123D;
		\item generuje ze vstupních dat grafy, které zobrazují vývoj koncentrace sledované látky na elementech;
		\item generuje mapy, které graficky znázorňují hodnoty koncentrací na daném řezu sítě;
		\item umožňuje vyexportovat grafy a mapy ve formátech rastrové i vektorové grafiky;
		\item umí zpracovat sadu úloh a vygenerovat požadované průběhy.
	\end{itemize}
\end{frame}

\begin{frame}{}{}
\begin{center}
\huge Děkuji za pozornost.
\end{center}
\end{frame}

\begin{frame}
	\frametitle{Otázky vedoucího práce -- 1.}
	\begin{block}{}
	{\footnotesize 	Na obrázku 8 uvádíte graf koncentrace v závislosti na čase. Zde jsou velice blízké koncentrace pro průměr, medián a 95\% kvantil. Element vlivem vzdálenosti od úložiště prakticky není zasažen kontaminací. Jak by se změnily výsledky na elementu v blízkosti povrchu a v blízkosti úložiště?}
	\end{block}
	\centering
		\includegraphics[width=0.5\paperwidth]{img/Melechov_sada_11122.pdf}\\
		{\tiny Obrázek 8 z textové části}
	\vfill
\end{frame}

\begin{frame}
	\frametitle{Otázky vedoucího práce -- 1.}
	\hspace*{-10mm}\makebox[\paperwidth][c]{%
		\begin{minipage}{0.5\paperwidth}
			\centering
			\includegraphics[width=0.5\paperwidth]{img/Melechov_sada_03779.pdf} \\
			{\tiny Element v blízkosti povrchu}
		\end{minipage}%
		\begin{minipage}{0.5\paperwidth}
			\centering
			\includegraphics[width=0.5\paperwidth]{img/Melechov_sada_19815.pdf} \\
			{\tiny Element v blízkosti úložiště}
		\end{minipage}
	}
\end{frame}

\begin{frame}
	\frametitle{Otázky vedoucího práce -- 2.}
	\begin{block}{}
		Software Flow123D umožňuje řešit transport i více látek najednou. Jaké operace byste musel vykonat, aby Váš software umožnil analyzovat taková data?
	\end{block}
	\begin{itemize}
		\item žádné změny ve struktuře aplikace nebo databáze -- aplikace počítá s možností tohoto rozšíření
		\item žádné změny v načítání vstupních dat
		\item upravení generování názvů výstupních souborů
		\item získání seznamu látek z databáze a provedení generování map a grafů pro všechny tyto látky
	\end{itemize}
	\vfill
\end{frame}

\begin{frame}
	\frametitle{Otázky vedoucího práce -- 3.}
	\begin{block}{}
		Při transportu více látek, jakým způsobem byste mohl rozšířit analýzy výsledků transportu?
	\end{block}
	\begin{itemize}
		\item pro každou z látek jsou dané hodnoty koncentrací [$ng/m^3$]
		\item na základě specifické aktivity dané látky určit objemovou aktivitu [$Bq/m^3$]
		\item po vynásobení konverzním faktorem lze určit objemovou dávku [$Sv/m^3$]
		\item objemové dávky všech látek lze sečíst a pro každý element vytvořit graf znázorňující vývoj celkové objemové dávky
		\item praktický význam: určení dopadu na lidský organismus po vypití jednoho~$m^3$ vody
	\end{itemize}
	\vfill
\end{frame}

\begin{frame}
	\frametitle{Otázky oponenta práce -- 1.}
	\begin{block}{}
		V kapitole 3.2 používáte jako argument proti knihovně NumPy nemožnost pracovat na řídkých datech. Nebylo by v tomto případě vhodnější použít balíček sparse z příbuzné knihovny SciPy než databázi SQLite?
	\end{block}
	\begin{itemize}
		\item stejná paměťová náročnost obou řešení
		\item práce s balíčkem sparse by byla o něco rychlejší, ale dopad na celkovou rychlost aplikace by byl minimální
		\item SQLite vhodnější pro ukládání vrcholů a~elementů -- snadné nalezení elementů, které se nachází v určité nadmořské výšce
		\item ideální řešení: SQLite pro ukládání dat o vrcholech a~elementech, balíček sparse pro ukládání hodnot koncentrací
	\end{itemize}
	\vfill
\end{frame}

\begin{frame}{}{}
\begin{center}
\huge Děkuji za pozornost.
\end{center}
\end{frame}

\end{document}